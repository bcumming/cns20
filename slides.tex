\documentclass[aspectratio=43]{beamer}
% Theme works only with a 4:3 aspect ratio
\usetheme{CSCS}

\usepackage{tikz}
\usepackage{pgfplots}
\usepackage{pgfplotstable}
\usetikzlibrary{pgfplots.groupplots,spy,patterns}
\usetikzlibrary{arrows.meta}
\usetikzlibrary{positioning}
\usepackage{listings}
\usepackage{color}
\usepackage{tcolorbox}
\usepackage{anyfontsize}
\usepackage{xspace}
\usepackage{graphicx}

% define footer text
\newcommand{\footlinetext}{Performance-portability: Arbor}

% Select the image for the title page
\newcommand{\picturetitle}{cscs_images/image5.pdf}

% fonts for maths
\usefonttheme{professionalfonts}
\usefonttheme{serif}

% source code listing
\newcommand\TS{\rule{0pt}{2.6ex}}       % Top strut
\newcommand\BS{\rule[-1.2ex]{0pt}{0pt}} % Bottom strut
\newcommand{\hl}[1]{\textbf{\textcolor{blue}{#1}}} % for hilighting optimal entries in tables
\newcommand{\rl}[1]{\textbf{\textcolor{red}{#1}}} % for hilighting sub-optimal entries in tables
\newcommand{\img}[1]{{\Large \textbf{IMAGE {#1}}}}
\newcommand{\hilight}[1]{\textcolor{blue!20!orange}{#1}}
\newcommand{\arbor}{{\ttfamily Arbor}\xspace}
\newcommand{\neuron}{{\ttfamily NEURON}\xspace}
\newcommand{\coreneuron}{{\ttfamily CoreNeuron}\xspace}
\newcommand{\daintmc}[0]{Daint-mc\xspace}
\newcommand{\daintgpu}[0]{Daint-gpu\xspace}
\newcommand{\tave}[0]{Tave-knl\xspace}
\newcommand{\pder}[2]{\frac{\partial{#1}}{\partial{#2}}}

% set indent to a more reasonable level (so that itemize can be used in columns)
\setlength{\leftmargini}{20pt}

\DeclareTextFontCommand{\emph}{\color{blue!85!black}}

\author{Ben Cumming}
\title{Portable simulation with \arbor.}
\subtitle{CNS*2020}
%\subtitle{CNS*2020 Workshop \emph{Tools and resources for developing and sharing models in computational neuroscience} \date{} }

\begin{document}

% TITLE SLIDE
\cscstitle

%-------------------------------------------
\begin{frame}[fragile]{}
    \begin{center}
        \includegraphics[height=0.15\textwidth]{logos/HBP_logo.jpg}
        \\ \vfill

        This research has received funding from the European Union’s Horizon 2020 Framework Programme for Research
        and Innovation under the Specific Grant Agreement No.  720270 (Human Brain Project SGA1), and Specific Grant
        Agreement No. 785907 (Human Brain Project SGA2).
        \\ \vfill

        \includegraphics[height=0.1\textwidth]{logos/julich_logo.pdf}
        \hspace{1cm}
        \includegraphics[height=0.09\textwidth]{logos/cscs_logo.pdf}
    \end{center}

\end{frame}
%-------------------------------------------

%-------------------------------------------
%\begin{frame}[fragile]{Overview}
%   Big Arbor logo

%   Overview of talk
%   \begin{enumerate}
%       \item What is \arbor and its motivation
%       \item Portability (performance)
%       \item Portability (model descriptions)
%       \item Allen model
%   \end{enumerate}
%end{frame}
%-------------------------------------------

%-------------------------------------------
\begin{frame}[fragile]{HPC Requires Portability}
    HPC is required to meet ambitious modeling aims.

    The \emph{free lunch} for exisiting software is over!
    \begin{itemize}
        \item Future gains in computational power will come from many-core processors (GPUs)
        \item All major HPC systems will be GPU based:
        \begin{itemize}
            \item Piz Daint @ CSCS (since 2015);
            \item Marconi100 @ Cineca (May 2020);
            \item EuroHPC pre-exascale systems (late 2021);
            \item US ECP pre-exascale and exascale systems (2021-2023)
        \end{itemize}
        \item \dots The ARM chips in Fugaku are closer to GPUs than a single core CPU.
    \end{itemize}
    Tools and models have to be portable across hardware architectures.
\end{frame}
%-------------------------------------------


%-------------------------------------------
\begin{frame}[fragile]{\arbor}
%-------------------------------------------
    \arbor is library for simulation of morphologically-detailed cells in large networks on HPC systems.
    \begin{itemize}
        \item \emph{key aim}: enabling simulation on all HPC systems.
        \item \emph{key aim}: provide rich interface enabling diverse use cases.
    \end{itemize}

    \vspace{10pt}
    \emph{All features} are implemented and optimised on \emph{all platforms}
    \begin{itemize}
        \item All GPUs (CUDA, HIP, Clang-CUDA)
        \item SIMD CPU backends: (AVX, AVX2, AVX512, Neon, SVE).
        \item Distributed simulation (MPI).
        \item Significantly lower memory and time requirements than \neuron and CoreNeuron.
    \end{itemize}
    %Requires a \emph{rich interface} for defining models.
    %\begin{itemize}
    %    \item Simulation of electrical current in arbitrarily complex morphologies.
    %    \item Arbitrary ion channel and synapse models.
    %    \item Inter-cell communication via spikes on arbitrary networks.
    %\end{itemize}
\end{frame}
%-------------------------------------------

%-------------------------------------------
\begin{frame}[fragile]{Models are varied and complicated}
% I like to use this classic image of a Purkinje cell to illustrate the challenges of
% developing a simulation tool like NEURON, CoreNEURON or Arbor.
% The comp. neuroscience community requires simulation tools that can efficiently simulate
% the different ways to represent this cell:
%       - as a leaky integrate and fire point neuron
%       - as a point neuron with user-supplied coupled odes describing state (e.g. HH)
%       - with few compartments and simple dynamics
%       - with detailed descriptions of every last branch in the arbor with spatial
%         variation between of ion channel distributions and stochastic synapses.
% Users also want the same flexibility in describing network dynamices, gap junctions,
% what variables to record for later analysis and so on.

    \begin{center}
        \includegraphics[width=0.45\textwidth]{images/purkinje_cell.png}
        \\
        {
            \scriptsize Drawing of a Purkinje cell in the cerebellar cortex by Santiago Ram\`{o}n y Caja.
        }
    \end{center}
\end{frame}
%-------------------------------------------

%-------------------------------------------
\begin{frame}[fragile]{\arbor}
% Key realisation: developing the efficient hardware optimised implementations of each
% of these model features is quite difficult. *HOWEVER* developing an interface that
% facilitates flexible description of models that can run on those back ends is much
% much harder, and must be integrated correctly with the underlying simulation engine.
    Efficient simulation of diverse models on different architectures requires focus on portability, interfaces and design.

    \vspace{30pt}

    The main challenge developing \arbor has been defining an interface for inputing models.
\end{frame}
%-------------------------------------------

%-------------------------------------------
%\begin{frame}[fragile]{\arbor}
%    It is hard to develop a portable tool that provides features and runs efficiently on the platforms we care about:
%    \begin{itemize}
%        \item design work for UI is hard and time consuming
%        \item proper sep. concerns in APIs hard
%        \item low level optimization hard
%    \end{itemize}
%
%    We have spent 5 years working on this problem
%    \begin{itemize}
%        \item one of the biggest challenges with writing a new tool that consumes complex models is model descriptions.
%        \item today I summarise some observations
%    \end{itemize}
%\end{frame}
%-------------------------------------------

%-------------------------------------------
\begin{frame}[fragile]{What is portability, anyway?}
% The challenge really boils down to what I call "portability"
Portability has two aspects:

\begin{enumerate}
\item \emph{Performance portability}, tools that:
    \begin{itemize}
        \item run efficiently and scale on different architectures;
        \item provide scale-agnostic, reproducable results;
        \item can be adapted to support new architectures.
    \end{itemize}

    \item \emph{Model portability}, model descriptions that are:
    \begin{itemize}
        \item simulator and architecture agnostic;
        \item flat (translated, not interpretted);
    \end{itemize}
\end{enumerate}

\end{frame}

\begin{frame}[fragile]{Portability}

These concerns are not orthogonal.
\begin{itemize}
    \item Require \emph{separation of concerns} between model description and lowered back end implemention.
\end{itemize}
\vspace{20pt}
Practically, this requires that model descriptions \emph{minimise}:
\begin{itemize}
    \item Architecture-specific or simulator-specific information;
    \item Arbitrary code (Python, HOC, SLI).
\end{itemize}

\end{frame}

\cscschapter{Separating the model from architecture}

% In this next section we will see how Arbor's API enforces this separation between
% model description and the target architecture on which the code will run.
%-------------------------------------------
\begin{frame}[fragile]{Step 1: Model abstraction (which model)}
    User models are described by a \emph{recipe}, which map cell \emph{gid} to:
    \begin{itemize}
        \item a description of the cell
        \begin{itemize}
            \item piecewise linear morphology
            \item named regions and locations
            \item ion channel and synapses
        \end{itemize}
        \item spike targets
        \item spike sources
        \item network connections that terminate on the cell
    \end{itemize}
    \vspace{5pt}
    Recipes provide a consistent flat description for all models.

    \vspace{5pt}
    Recipe descriptions are functional, enabling lazy evaluation for efficient parallel model construction.

    \vspace{5pt}
    Recipes contain no hardware or implementation details.
\end{frame}
%-------------------------------------------

%-------------------------------------------
\begin{frame}[fragile]{Step 2: Hardware context (which hardware)}

    \vspace{-20pt}

    \begin{code}{Select hardware resources}
        \begin{lstlisting}[style=talkpython]
import arbor
from mpi4py import MPI

rec = my_recipe() # user defined model
ctx = arbor.context(threads=12, gpu_id=0, mpi=MPI.COMM_WORLD)
        \end{lstlisting}
    \end{code}

\vfill

    Users can select hardware resources at run time:
    \begin{itemize}
        \item Number of threads in thread pool
        \item Which GPU [optional]
        \item Which MPI communicator [optional]
    \end{itemize}
\end{frame}
%-------------------------------------------

%-------------------------------------------
\begin{frame}[fragile]{Step 3: Domain decomposition}
    \begin{code}{Define domain decomposition}
        \begin{lstlisting}[style=talkpython]
import arbor
from mpi4py import MPI

rec = my_recipe() # user defined model
ctx = arbor.context(threads=12, gpu_id=0, mpi=MPI.COMM_WORLD)
dst = arbor.partition_load_balance(rec, ctx)
        \end{lstlisting}
    \end{code}

    Which cells to compute where:
    \begin{enumerate}
        \item Assign each cell to an MPI rank
        \item Assign cells into groups on each rank to CPU and GPU resources
    \end{enumerate}

    \vspace{10pt}

    Caller can optionally provide hard-coded decomposition.

    \vfill
\end{frame}
%-------------------------------------------

%-------------------------------------------
\begin{frame}[fragile]{Step 4: Instantiate model}
    \begin{code}{Instantiate model on target compute resources}
        \begin{lstlisting}[style=talkpython]
import arbor
from mpi4py import MPI

rec = my_recipe() # user defined model
ctx = arbor.context(threads=12, gpu_id=0, mpi=MPI.COMM_WORLD)
sim = arbor.simulation(rec, ctx)
        \end{lstlisting}
    \end{code}

    A simulation object:
    \begin{itemize}
        \item Instantiates target-specific data structures and call backs
        \item Provides a generic interface for:
        \begin{itemize}
            \item steering simulation
            \item sampling spikes, voltages, etc.
        \end{itemize}
        \item Has no global state
        \begin{itemize}
            \item multiple simulations can be instantiated simultaneously.
        \end{itemize}
    \end{itemize}

    Caller can optionally provide hints on how to assign model to hardware resources.
\end{frame}
%-------------------------------------------

%-------------------------------------------
%\begin{frame}[fragile]{It is APIs all the way down, young man!}
%    Components communicate via APIs allow that allow implementation of new cell models, communication methods, hardware back ends etc.
%    \begin{center}
%        \includegraphics[width=0.8\textwidth]{images/api.pdf}
%    \end{center}
%    Adding a new hardware back end does not touch a single line of simulation state or front end code.
%\end{frame}
%-------------------------------------------

\cscschapter{Describing Single Cells}

%-------------------------------------------
\begin{frame}[fragile]{}
    Constructing descriptions of single cells is the main portability challenge with NEURON models
    \begin{enumerate}
        \item The NMODL language describes dynamics of individual mechanisms.
        \item Distribution and properties of mechanisms with a Turing complete language
              like Python/HOC is non-portable.
    \end{enumerate}

    Models need to describe locations and regions on the morphology:
    \begin{itemize}
        \item synapse sites
        \item distribution of ion channels
    \end{itemize}

    NeuroML is an example of a flat description.
    \begin{itemize}
        \item Arbor uses its own DSL for regions and locations.
    \end{itemize}
\end{frame}
%-------------------------------------------

%-------------------------------------------
\begin{frame}[fragile]{Step 1: Morphology Representation}
    Cells are composed of \emph{cable segments} that:
    \begin{itemize}
        \item are truncated conic frustums;
        \item can be dislocated;
        \item form branches (c.f. SONATA sections).
    \end{itemize}

    \begin{center}
        \includegraphics[width=\textwidth]{./morphos/morphlab.png}
        \\
        \small \textit{(Left) Segments (red=soma, grey=axon, blue=dend);\\(right) The branches.}
    \end{center}

    Can represent SWC, NeuroML and Neurolicida formats.
    \begin{itemize}
        \item \texttt{<rant>} \textit{please} don't use ``spherical'' somas in SWC files: it makes no sense. \texttt{</rant>}
    \end{itemize}

\end{frame}
%-------------------------------------------

%-------------------------------------------
\begin{frame}[fragile]{Step 2: Define regions and locations}

    A \emph{locset} is a multiset of locations on a morphology, e.g.:
    \begin{itemize}
        \item The center of the soma.
        \item The locations of inhibitory synapses.
        \item The tips of the dendritic arbor.
    \end{itemize}
    \vspace{5pt}

    A \emph{region} is a subset of a morphology's cable segments, e.g.:
    \begin{itemize}
        \item The soma.
        \item The dendritic arbor.
        \item The distal half of branch 1.
        \item Everywhere greater than 100 um from the soma.
        \item The dendrites with radius less than 1 um.
    \end{itemize}
    \vspace{5pt}
    \arbor provides a DSL for their succinct description.

\end{frame}
%-------------------------------------------

%-------------------------------------------
\begin{frame}[fragile]{The region and locset DSL}
    The DSL uses \emph{s-expressions} and is composable:

    \begin{lstlisting}[style=arblang]
(distal_interval             ; take subtrees that start at
  (proximal                  ; locations closest to the soma
    (radius_lte              ; with radius <= 0.2 um
      (join (tag 3) (tag 4)) ; on basal and apical dendrites
      0.2)))
    \end{lstlisting}

    Expressions are given labels in a dictionary:

    \begin{lstlisting}[style=talkpython]
labels = {'soma': '(tag 1)',
          'axon': '(tag 2)',
          'dend': '(tag 3)',
          'apic': '(tag 4)',
          'inhib': '(uniform (region "dend")   0  99 0)',
          'stim-site': '(cable 0 0.5)'}
    \end{lstlisting}

    Replace \& simplify non-portable and bug-prone hoc templates.
    \begin{itemize}
        \item Describes what, not how.
        \item Not part of the morphology description.
    \end{itemize}

\end{frame}
%-------------------------------------------

%-------------------------------------------
\begin{frame}[fragile]{Locset examples}
    \begin{columns}[T]
        \begin{column}{0.5\textwidth}
            \includegraphics[width=\textwidth]{images/ls_term.png}

            \vspace{10pt}

            \begin{lstlisting}[style=arblang]
(restrict (terminal) (tag 3))
            \end{lstlisting}

            The tips (terminals) of the dendritic arbor (tag 3).

        \end{column}
        \begin{column}{0.5\textwidth}
            \includegraphics[width=\textwidth]{images/ls_uniform.png}

            \vspace{10pt}

            \begin{lstlisting}[style=arblang]
(uniform (tag 3) 0 9 0)
            \end{lstlisting}

            Ten random locations on the dendritic arbor.

        \end{column}
    \end{columns}
\end{frame}
%-------------------------------------------

%-------------------------------------------
\begin{frame}[fragile]{Region examples}
    \begin{columns}[T]
        \begin{column}{0.5\textwidth}
            \includegraphics[width=\textwidth]{images/reg_tag3.png}

            \vspace{10pt}

            \begin{lstlisting}[style=arblang]
(tag 3)
            \end{lstlisting}

            The dendritic arbor, which has tag 3, derived from the SWC structure identifier.

        \end{column}
        \begin{column}{0.5\textwidth}
            \includegraphics[width=\textwidth]{images/reg_radle5.png}

            \vspace{10pt}

            \begin{lstlisting}[style=arblang]
(radius_le (all) 0.5)
            \end{lstlisting}
            All parts of the cell with radius less than or equal to 0.5 $\mu$m.

        \end{column}
    \end{columns}
\end{frame}
%-------------------------------------------

%-------------------------------------------
\begin{frame}[fragile]{Step 3: Apply to a morphology}
    A label dictionary is combined with a morphology to build a concretised set of 
    regions and locsets

    \vspace{20pt}

    Python code.
\end{frame}
%-------------------------------------------

%-------------------------------------------
\begin{frame}[fragile]{Step 4: Decorate the cell}
    The regions are used to 
\end{frame}
%-------------------------------------------

\cscschapter{Allen}

%-------------------------------------------
\begin{frame}[fragile]{}
    For this talk we took model from Allen DB

    What is Allen DB

    Aim to test how "portable" SONATA is

        If we can parse it without opening NEURON or reverse engineering the AllenSDK then it is portbale.
\end{frame}
%-------------------------------------------

%-------------------------------------------
\begin{frame}[fragile]{Allen}
    Ported NMODL

        Now available in Arbor

    Morphology was a pain point

        SDK performs undocumented modifications to axons in *more than one location*

        Spherical somas don't exist in NEURON, and we deprecated them in Arbor based on this experience
        (they are a bad idea because they require "interpretation" of the input)

mechanism naming, and layout of JSON input files reflected NEURON semantics and were not structured enough.

    We needed a separate version of the runner script for each cell (differences in json layout between models)
\end{frame}
%-------------------------------------------

%-------------------------------------------
\begin{frame}[fragile]{}
    Results: cell model name

    Plot comparing voltage traces

    Timing information
\end{frame}
%-------------------------------------------

\cscschapter{Conclusion}

%-------------------------------------------
\begin{frame}[fragile]{Retrospective}
    After 5 years of development we have come to realise that...
    \begin{itemize}
        \item The number one problem stopping adoption of new tools is non-portable model descitions:
        \begin{itemize}
            \item 
        \end{itemize}
        \item The amount of time spent hacking on low level GPU or vectorized code is much less than the time spent on interfaces.
        \item Unit tests pay you back more the longer you have them.
    \end{itemize}
    \vspace{10pt}
    SONATA is a promising move towards portability.
\end{frame}
%-------------------------------------------

%-------------------------------------------
\begin{frame}[fragile]{}
    \arbor is under active, open, development.

    \vspace{10pt}

    \begin{center}
        \arbor is \emph{open source software}:\\
        \vspace{3pt}
        \begin{lstlisting}[style=talkpseudo]
                  github.com/arbor-sim/arbor
        \end{lstlisting}
    \end{center}
\end{frame}
%-------------------------------------------

\end{document}
